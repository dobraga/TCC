%% abtex2-modelo-relatorio-tecnico.tex, v<VERSION> laurocesar
%% Copyright 2012-<COPYRIGHT_YEAR> by abnTeX2 group at http://www.abntex.net.br/ 
%%
%% This work may be distributed and/or modified under the
%% conditions of the LaTeX Project Public License, either version 1.3
%% of this license or (at your option) any later version.
%% The latest version of this license is in
%%   http://www.latex-project.org/lppl.txt
%% and version 1.3 or later is part of all distributions of LaTeX
%% version 2005/12/01 or later.
%%
%% This work has the LPPL maintenance status `maintained'.
%% 
%% The Current Maintainer of this work is the abnTeX2 team, led
%% by Lauro César Araujo. Further information are available on 
%% http://www.abntex.net.br/
%%
%% This work consists of the files abntex2-modelo-relatorio-tecnico.tex,
%% abntex2-modelo-include-comandos and abntex2-modelo-references.bib
%%

% ------------------------------------------------------------------------
% ------------------------------------------------------------------------
% abnTeX2: Modelo de Relatório Técnico/Acadêmico em conformidade com 
% ABNT NBR 10719:2015 Informação e documentação - Relatório técnico e/ou
% científico - Apresentação
% ------------------------------------------------------------------------ 
% ------------------------------------------------------------------------

\documentclass[
	% -- opções da classe memoir --
	12pt,				% tamanho da fonte
	a4paper,		% tamanho do papel. 
	oneside,    % remover paginas em branco
	% -- opções da classe abntex2 --
	chapter=TITLE,		   % títulos de capítulos convertidos em letras maiúsculas
	section=TITLE,		   % títulos de seções convertidos em letras maiúsculas
	subsection=TITLE,	   % títulos de subseções convertidos em letras maiúsculas
	subsubsection=TITLE, % títulos de subsubseções convertidos em letras maiúsculas
	% -- opções do pacote babel --
	english,			% idioma adicional para hifenização
	french,				% idioma adicional para hifenização
	spanish,			% idioma adicional para hifenização
	brazil,				% o último idioma é o principal do documento
]{abntex2}


% ---
% PACOTES
% ---

%--
% Modificações
\usepackage{./latex/ence}

% ---
% Pacotes fundamentais 
% ---
%\usepackage{lmodern}			% Usa a fonte Latin Modern
\usepackage[T1]{fontenc}		% Selecao de codigos de fonte.
\usepackage[utf8]{inputenc}		% Codificacao do documento (conversão automática dos acentos)
\usepackage{indentfirst}		% Indenta o primeiro parágrafo de cada seção.
\usepackage{color}				% Controle das cores
\usepackage{graphicx}			% Inclusão de gráficos
\usepackage{microtype} 			% para melhorias de justificação
% ---

% ---
% Pacotes adicionais, usados no anexo do modelo de folha de identificação
% ---
\usepackage{multicol}
\usepackage{multirow}
% ---

% ---
% Pacotes de citações
% ---
\usepackage[brazilian,hyperpageref]{backref}	 % Paginas com as citações na bibl
\usepackage[alf]{abntex2cite}	% Citações padrão ABNT

% ---
% Pacote para que imagens e tabelas não mudem de lugar
% ---
\usepackage{float}
\floatplacement{figure}{H}
\floatplacement{table}{H}

% ---
% Informações de dados para CAPA e FOLHA DE ROSTO
% ---
\titulo{Optimização Bayesiana}
\autor{Douglas Martins Mendes Braga\\
Felipe Sales}
\local{Rio de Janeiro}
\data{2020}
\instituicao{Instituto Brasileiro de Geografia e Estatística - IBGE

Escola Nacional de Ciências Estatísticas - ENCE

Bacharelado em Estatística}
\orientador{aaaaaaaaaa}
\coorientador{lalalala}
\preambulo{Monografia apresentada à Escola Nacional de Ciências Estatísticas do
Instituto Brasileiro de Geografia e Estatística como requisito parcial à
obtenção do título de Bacharel em Estatística.}
% ---

% ---
% Configurações de aparência do PDF final

% informações do PDF
\makeatletter
\hypersetup{
	  colorlinks=false,      		% false: boxed links; true: colored links
	  bookmarksdepth=4
}
\makeatother
% --- 

% --- 
% Espaçamentos entre linhas e parágrafos 
% --- 

% O tamanho do parágrafo é dado por:
\setlength{\parindent}{1.3cm}

% Controle do espaçamento entre um parágrafo e outro:
\setlength{\parskip}{0.2cm}  % tente também \onelineskip

% Seleciona o idioma do documento (conforme pacotes do babel)
\selectlanguage{brazil}

% Retira espaço extra obsoleto entre as frases.
\frenchspacing 

%\usepackage{fontspec}
%\setmainfont[Ligatures=TeX]{Calibri}

%
\usepackage{caption}
\captionsetup{%
    %,format=hang
    ,justification=centering
    %,singlelinecheck=false
    ,figureposition=top
    ,position=top
    }

% ----
% Início do documento
% ----
\begin{document}

% ----------------------------------------------------------
% ELEMENTOS PRÉ-TEXTUAIS
% ----------------------------------------------------------
\pretextual

% ---
% compila o indice
% ---
\makeindex
% ---

% ---
% Capa
% ---
\imprimircapa
% ---

% ---
% Folha de rosto
% (o * indica que haverá a ficha bibliográfica)
% ---
\imprimirfolhaderosto*


% ------------------
% Folha de aprovação
%-------------------
\imprimitfolhadeaprovacao

% ---
% Agradecimentos
% ---
\begin{agradecimentos}
  O agradecimento principal é direcionado a Youssef Cherem.
  
  Os agradecimentos especiais são direcionados ao Centro de Pesquisa em
  Arquitetura da Informação\footnote{\url{http://www.cpai.unb.br/}} da
  Universidade de Brasília (CPAI), ao grupo de usuários
  \emph{latex-br}\footnote{\url{http://groups.google.com/group/latex-br}}
  e aos novos voluntários do grupo
  \emph{\abnTeX}\footnote{\url{http://groups.google.com/group/abntex2} e
  \url{http://www.abntex.net.br/}}\textasciitilde{}que contribuíram e que
  ainda contribuirão para a evolução do abn\TeX.
\end{agradecimentos}
% ---

% ---
% RESUMO
% ---

% resumo na língua vernácula (obrigatório)
\setlength{\absparsep}{18pt} % ajusta o espaçamento dos parágrafos do resumo
\begin{resumo}
 O resumo deve ressaltar o objetivo, o método, os resultados e as
 conclusões do documento. A ordem e a extensão destes itens dependem do
 tipo de resumo (informativo ou indicativo) e do tratamento que cada item
 recebe no documento original. O resumo deve ser precedido da referência
 do documento, com exceção do resumo inserido no próprio documento.
 (\ldots) As palavras-chave devem figurar logo abaixo do resumo,
 antecedidas da expressão Palavras-chave:, separadas entre si por ponto e
 finalizadas também por ponto.
 \noindent
 \\
 \\
 \textbf{Palavras-chaves}: Otimização. Estatística Bayesiana. Machine Learning.
\end{resumo}
% ---

% ---
% inserir lista de ilustrações
% ---
\pdfbookmark[0]{\listfigurename}{lof}
\listoffigures*
\clearpage
% ---

% ---
% inserir lista de tabelas
% ---
\pdfbookmark[0]{\listtablename}{lot}
\listoftables*
\clearpage
% ---

% ---
% inserir o sumario
% ---
\pdfbookmark[0]{\contentsname}{toc}
\tableofcontents*
\clearpage
% ---

% ----------------------------------------------------------
% ELEMENTOS TEXTUAIS
% ----------------------------------------------------------
\textual
%\pagestyle{plain}

\hypertarget{introduuxe7uxe3o}{%
\chapter{Introdução}\label{introduuxe7uxe3o}}

Os algoritmos de Machine Learning raramente não possuem parãmetros, como
por exemplo, a taxa de aprendizado. Muitas vezes esses parâmetros são
definidos por tentativa e erro.

A optimização bayesiana é um método para encontrar pontos próximos a
pontos ótimos das funções, esta técnica utiliza-se de suposições a
priori e combina com evidências dos dados para obter a posteriori. ~

\begin{table}

\caption{\label{tab:unnamed-chunk-5}Iris}
\centering
\begin{tabular}[t]{r|r|r|r|l}
\hline
Sepal.Length & Sepal.Width & Petal.Length & Petal.Width & Species\\
\hline
5.1 & 3.5 & 1.4 & 0.2 & setosa\\
\hline
4.9 & 3.0 & 1.4 & 0.2 & setosa\\
\hline
4.7 & 3.2 & 1.3 & 0.2 & setosa\\
\hline
4.6 & 3.1 & 1.5 & 0.2 & setosa\\
\hline
5.0 & 3.6 & 1.4 & 0.2 & setosa\\
\hline
\multicolumn{5}{l}{\textit{Fonte: } O próprio autor}\\
\end{tabular}
\end{table}

\#\texttt{\{r\ child=\textquotesingle{}02-revisao.rmd\textquotesingle{}\}\ \#}

\#\texttt{\{r\ child=\textquotesingle{}03-dados.rmd\textquotesingle{}\}\ \#}

\#\texttt{\{r\ child=\textquotesingle{}04.0-metodologia.rmd\textquotesingle{}\}\ \#}

\newpage

\hypertarget{referuxeancias}{%
\chapter{REFERÊNCIAS}\label{referuxeancias}}

% ----------------------------------------------------------
% ELEMENTOS PÓS-TEXTUAIS
% ----------------------------------------------------------
\postextual

% ----------------------------------------------------------
% Referências bibliográficas
% ----------------------------------------------------------
\bibliography{bibliography.bibtex}

\printindex

\end{document}
